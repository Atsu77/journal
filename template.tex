%% template for IEICE Transactions
%% v2.1 [2015/10/31]
\documentclass[paper]{ieice}
%\documentclass[invited]{ieice}
%\documentclass[position]{ieice}
%\documentclass[survey]{ieice}
%\documentclass[invitedsurvey]{ieice}
%\documentclass[review]{ieice}
%\documentclass[tutorial]{ieice}
%\documentclass[letter]{ieice}
%\documentclass[brief]{ieice}
%\usepackage[dvips]{graphicx}
%\usepackage[pdftex]{graphicx,xcolor}
\usepackage[dvipdfmx]{graphicx,xcolor}
\usepackage[fleqn]{amsmath}
\usepackage{newtxtext}
\usepackage[varg]{newtxmath}
%
% 数式の表記に利用
\newcommand{\abs}[1]{\left\lvert#1\right\rvert}
\DeclareMathOperator*{\E}{E}
\DeclareMathOperator*{\SD}{SD}
\DeclareMathOperator*{\var}{var}
%
\graphicspath{{./figure/}}
%
\setcounter{page}{1}
%\breakauthorline{}% breaks lines after the n-th author

\field{}
%\SpecialIssue{}
%\SpecialSection{}
%\theme{}
\title[Road Obstacle Detection by Autoencoder]{Road Obstacle Detection by Autoencoder Using Vehicle Driving Information}
%\title[title for header]{title}
%\titlenote{}
\authorlist{%
%
 \authorentry{Atsuhide YAMANE}{}{}\MembershipNumber{}
 \authorentry{Masahiro FUJII}{}{}\MembershipNumber{}
% \authorentry{name}{membership}{affiliate label}\MembershipNumber{}
% \authorentry{name}{membership}{affiliate label}[present affiliate label]\MembershipNumber{}
% \authorentry[e-mail address]{name}{membership}{affiliate label}\MembershipNumber{}
% \authorentry[e-mail address]{name}{membership}{affiliate label}[present affiliate label]\MembershipNumber{}
}
\affiliate[affiliate label]{The author is with the
}
%\paffiliate[present affiliate label]{Presently, the author is with the }

\received{2015}{1}{1}
\revised{2015}{1}{1}

%% <local definitions here>

%% </local definitions here>

\begin{document}
\maketitle

\begin{summary}

\end{summary}
\begin{keywords}
  ITS, ETC, Obstacle detection, Unsupervised learning
\end{keywords}

\section{Introduction}
\label{sec:introduction}
%
% 近年,ETC (Electronic Toll Collection System) 2.0と呼ばれる,道路上に設置されるITS (Intelligent Transport Systems) スポットを通じ様々な情報サービスを提供するシステムが普及している.
In recent years, ETC (Electronic Toll Collection system) 2.0, which provides various information services through ITS (Intelligent Transport Systems) spots installed on highways, has become popular.
%
% ETC2.0は,車両の位置や加速度,車線変更情報などの車両走行情報を取得し,全国の高速道路上のITSスポットに送信する.
ETC2.0 collects vehicle driving information, such as vehicle position, acceleration, and lane change information, and transmits it to ITS spots on highways throughout Japan.
%
% ITSスポットとは,道路交通情報を提供するサービスであり,従来使用されてきた光ビーコンや電波ビーコンにとって代わる路側機である\cite{itsspot}.
ITS spot is a service that provides road traffic information and is a roadside device that replaces the optical and radio beacons that have been used in the past\cite{itsspot}.
%
% 図\ref{fig:itsspot_location}に示すように,日本全国の高速道路に設置されており\cite{etc2.0Location},都市内の高速道路では約4 km間隔,都市間の高速道路では約10から15 km間隔で設置されている.
As shown in Figure \ref{fig:itsspot_location}, they are installed on highways throughout Japan at intervals of about 4 km on inner-city highways and about 10 to 15 km on inter-city highways \cite{etc2.0Location}.
%
\begin{figure}[tb]
  \begin{center}
    \includegraphics[width=1.0\linewidth]{itsspot_location.pdf}
  \end{center}
  \caption{Location of ITS spots on highways in Japan}
  \label{fig:itsspot_location}
\end{figure}
%
% ETC2.0からITSスポットに送信された車両走行情報に基づいて,道路の状況を把握し,ドライバーに運転支援のための様々なサービスを提供する.
Based on vehicle driving information transmitted from ETC2.0 to ITS spots, the system monitors road conditions and provides various driving assistance services to the driver.
%
\par
%
% ETC2.0から提供されるサービスとして路上障害物情報がある.
One of the services provided by ETC2.0 is road obstacle information.
%
% 2018年の高速道路での落下物処理件数は31.6万件\cite{obstacles}であり,これは約1分間に1件発生している計算となり,多くのドライバーの安全運転の妨げとなっている.
In 2018, 316,000 were handled on Japan's highways \cite{obstacles}, which translates to about one occurrence every minute, hindering the safe driving of many drivers.
%
% 路上障害物の撤去にはその位置情報が必要になり,現在はドライバーやパトロールの通報によって情報が提供されているため検知のリアルタイム性の確保が困難になっている.
It is to need information on their location for removing road obstacles.
%
Because this information is currently provided by driver and patrol reports, it is difficult to ensure real-time detection.
%
\par
%
% 検知の自動化を行うため,過去の車載カメラ映像との道路面の差分により障害物検知を行う手法\cite{hisatoku}や,物体検知を行うモデルの1つであるYOLOv3を用いて,障害物の位置や種類を特定する手法\cite{wang2021front}が提案されている.
The authors proposed two methods to automate obstacle detection.
%
The first method relies on the difference of the road surface from previous in-vehicle camera images \cite{hisatoku}, whereas the second one utilizes YOLOv3, an object detection model \cite{wang2021front}, to identify the location and type of obstacles.
%
% しかし,これらの手法はカメラを使用しており悪天候時や夜間では検知性能が低下することが懸念される.
However, these methods use cameras, and there is concern that detection performance may deteriorate in bad weather or at night.
%
\par
%
% 一方で車両走行情報は,天候や時間帯に依存しないため,検知性能の低下を防ぐことができると考えられる.
Vehicle driving information is independent of weather and time of day.
%
This is expected to prevent the degradation of detection performance.
%
% そこで,本稿でETC2.0とITSスポットを介して取得された車両走行情報を用いて障害物検知を行う手法を提案し,既存手法と提案手法の検知性能を比較する.
Therefore, we propose a method for obstacle detection using vehicle driving information collected via ETC2.0 and ITS spots.
%
\section{Traffic flow model}
\label{sec:traffic_flow_model}
%
% 障害物検知のシステムの実用化のためには,実環境での検証が必要である.
To put a road obstacle detection system into practical use, it is necessary to verify the system in a real environment.
%
% しかし,路上障害物を実道路上に放置することを想定する実証検証は実験のための場所や設備の確保,費用が多大になり,実現が困難である.
However, it is difficult to realize a verification test that assumes that roadside obstacles are left on the actual road because of the large cost of securing a location and facilities for the test.
%
% そこで,本研究では交通流シミュレータ SUMO (Simulation of Urban Mobility)を使用して検証を行う.
Therefore, in this thesis, the traffic flow simulator SUMO (Simulation of Urban Mobility) \cite{sumo} is used for verification.
%
% SUMOとはドイツ航空宇宙センターの輸送システム研究所で作成されたオープンソースのシミュレータであり,各車両の減速値や車線変更などの車両走行情報を取得できる.
SUMO is an open-source simulator created by the Institute of Transportation Systems at the German Aerospace Center.
%
\subsection{Road model}
\label{sec:road_model}
%
% 本研究では,ITSスポットが設置されている高速道路を想定し,道路全長4000 m,片側3車線の一方通行のモデルとし,その間の1000 mを評価区間とした.
This thesis used a one-way model with a total road length of 4000 m and three lanes in one way, assuming an expressway with ITS spots installed as shown in the figure.
%
% 実環境に近い検証を行うため,モデルは道路幅は車線ごとに3.5 m,制限速度を80 〜 100 km/hに設定し,左車線を最優先走行車線とした.
To verify the model as it would be in a realistic environment, the road width was set to 3.5 m for each lane, the speed limit was set to 80 to 100 km/h, and the left lane was set as the highest priority lane.
%
\subsection{Vehicle model}
\label{sec:vehicle_model}
%
% 車両モデルは,主にITSスポットが設置されている高速道路の車種区分\cite{syasyukubun}に基づいて設定した.
The vehicle model was set up based mainly on the vehicle type classification of the expressway on which the ITS spots are installed \cite{syasyukubun}.
%
% 各車両を全長3.5 m,幅1.4 m,最高速度100 km/hと設定し,道路の0 m地点から毎秒1台の車両が任意の車線から発生するようにした.
Each vehicle was assumed to be 3.5 m long, have a maximum speed of 100 km/h, and was generated from the 0 m point on the road.
%
% 前提として,すべての車両はETC2.0車載器を搭載し,評価区間内のすべての車両の車両走行情報を取得できるものとする.
All vehicles are assumed to be equipped with ETC2.0 onboard equipment capable of acquiring vehicle travel information.
%


\bibliographystyle{ieicetr}% bib style
\bibliography{myRef}% your bib database
\begin{thebibliography}{99}% more than 9 --> 99 / less than 10 --> 9
\bibitem{}
\end{thebibliography}

%\profile{}{}
%\profile*{}{}% without picture of author's face

\end{document}
