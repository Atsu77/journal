%% template for IEICE Transactions
%% v2.1 [2015/10/31]
\documentclass[paper]{ieice}
%\documentclass[invited]{ieice}
%\documentclass[position]{ieice}
%\documentclass[survey]{ieice}
%\documentclass[invitedsurvey]{ieice}
%\documentclass[review]{ieice}
%\documentclass[tutorial]{ieice}
%\documentclass[letter]{ieice}
%\documentclass[brief]{ieice}
%\usepackage[dvips]{graphicx}
%\usepackage[pdftex]{graphicx,xcolor}
\usepackage[dvipdfmx]{graphicx,xcolor}
\usepackage[fleqn]{amsmath}
\usepackage{newtxtext}
\usepackage[varg]{newtxmath}
%
% 数式の表記に利用
\newcommand{\abs}[1]{\left\lvert#1\right\rvert}
\DeclareMathOperator*{\E}{E}
\DeclareMathOperator*{\SD}{SD}
\DeclareMathOperator*{\var}{var}
%
\graphicspath{{./figure/}}
%
\setcounter{page}{1}
%\breakauthorline{}% breaks lines after the n-th author

\field{}
%\SpecialIssue{}
%\SpecialSection{}
%\theme{}
\title[Road Obstacle Detection by Autoencoder]{Road Obstacle Detection by Autoencoder Using Vehicle Driving Information}
%\title[title for header]{title}
%\titlenote{}
\authorlist{%
%
 \authorentry{Atsuhide YAMANE}{}{}\MembershipNumber{}
 \authorentry{Masahiro FUJII}{}{}\MembershipNumber{}
% \authorentry{name}{membership}{affiliate label}\MembershipNumber{}
% \authorentry{name}{membership}{affiliate label}[present affiliate label]\MembershipNumber{}
% \authorentry[e-mail address]{name}{membership}{affiliate label}\MembershipNumber{}
% \authorentry[e-mail address]{name}{membership}{affiliate label}[present affiliate label]\MembershipNumber{}
}
\affiliate[affiliate label]{The author is with the
}
%\paffiliate[present affiliate label]{Presently, the author is with the }

\received{2015}{1}{1}
\revised{2015}{1}{1}

%% <local definitions here>

%% </local definitions here>

\begin{document}
\maketitle

\begin{summary}

\end{summary}
\begin{keywords}
  ITS, ETC, Obstacle detection, Unsupervised learning
\end{keywords}

\section{Introduction}
\label{sec:introduction}
%
% 近年,ETC (Electronic Toll Collection System) 2.0と呼ばれる,道路上に設置されるITS (Intelligent Transport Systems) スポットを通じ様々な情報サービスを提供するシステムが普及している.
In recent years, ETC (Electronic Toll Collection system) 2.0, which provides various information services through ITS (Intelligent Transport Systems) spots installed on highways, has become popular.
%
% ETC2.0は,車両の位置や加速度,車線変更情報などの車両走行情報を取得し,全国の高速道路上のITSスポットに送信する.
ETC2.0 collects vehicle driving information, such as vehicle position, acceleration, and lane change information, and transmits it to ITS spots on highways throughout Japan.
%
% ITSスポットとは,道路交通情報を提供するサービスであり,従来使用されてきた光ビーコンや電波ビーコンにとって代わる路側機である\cite{itsspot}.
ITS spot is a service that provides road traffic information and is a roadside device that replaces the optical and radio beacons that have been used in the past\cite{itsspot}.
%
% 図\ref{fig:itsspot_location}に示すように,日本全国の高速道路に設置されており\cite{etc2.0Location},都市内の高速道路では約4 km間隔,都市間の高速道路では約10から15 km間隔で設置されている.
As shown in Figure \ref{fig:itsspot_location}, they are installed on highways throughout Japan at intervals of about 4 km on inner-city highways and about 10 to 15 km on inter-city highways \cite{etc2.0Location}.
%
\begin{figure}[tb]
  \begin{center}
    \includegraphics[width=1.0\linewidth]{itsspot_location.pdf}
  \end{center}
  \caption{Location of ITS spots on highways in Japan}
  \label{fig:itsspot_location}
\end{figure}
%
% ETC2.0からITSスポットに送信された車両走行情報に基づいて,道路の状況を把握し,ドライバーに運転支援のための様々なサービスを提供する.
Based on vehicle driving information transmitted from ETC2.0 to ITS spots, the system monitors road conditions and provides various driving assistance services to the driver.
%
\par
%
% ETC2.0から提供されるサービスとして路上障害物情報がある.
One of the services provided by ETC2.0 is road obstacle information.
%
% 2018年の高速道路での落下物処理件数は31.6万件\cite{obstacles}であり,これは約1分間に1件発生している計算となり,多くのドライバーの安全運転の妨げとなっている.
In 2018, 316,000 were handled on Japan's highways \cite{obstacles}, which translates to about one occurrence every minute, hindering the safe driving of many drivers.
%
% 路上障害物の撤去にはその位置情報が必要になり,現在はドライバーやパトロールの通報によって情報が提供されているため検知のリアルタイム性の確保が困難になっている.
It is to need information on their location for removing road obstacles.
%
Because this information is currently provided by driver and patrol reports, it is difficult to ensure real-time detection.
%
\par
%
% 検知の自動化を行うため,過去の車載カメラ映像との道路面の差分により障害物検知を行う手法\cite{hisatoku}や,物体検知を行うモデルの1つであるYOLOv3を用いて,障害物の位置や種類を特定する手法\cite{wang2021front}が提案されている.
The authors proposed two methods to automate obstacle detection.
%
The first method relies on the difference of the road surface from previous in-vehicle camera images \cite{hisatoku}, whereas the second one utilizes YOLOv3, an object detection model \cite{wang2021front}, to identify the location and type of obstacles.
%
% しかし,これらの手法はカメラを使用しており悪天候時や夜間では検知性能が低下することが懸念される.
However, these methods use cameras, and there is concern that detection performance may deteriorate in bad weather or at night.
%
\par
%
% 一方で車両走行情報は,天候や時間帯に依存しないため,検知性能の低下を防ぐことができると考えられる.
Vehicle driving information is independent of weather and time of day.
%
This is expected to prevent the degradation of detection performance.
%
% そこで,本稿でETC2.0とITSスポットを介して取得された車両走行情報を用いて障害物検知を行う手法を提案し,既存手法と提案手法の検知性能を比較する.
Therefore, we propose a method for obstacle detection using vehicle driving information collected via ETC2.0 and ITS spots.
%
\section{Traffic flow model}
\label{sec:traffic_flow_model}
%
% 障害物検知のシステムの実用化のためには,実環境での検証が必要である.
To put a road obstacle detection system into practical use, it is necessary to verify the system in a real environment.
%
% しかし,路上障害物を実道路上に放置することを想定する実証検証は実験のための場所や設備の確保,費用が多大になり,実現が困難である.
However, it is difficult to realize a verification test that assumes that roadside obstacles are left on the actual road because of the large cost of securing a location and facilities for the test.
%
% そこで,本研究では交通流シミュレータ SUMO (Simulation of Urban Mobility)を使用して検証を行う.
Therefore, in this thesis, the traffic flow simulator SUMO (Simulation of Urban Mobility) \cite{sumo} is used for verification.
%
% SUMOとはドイツ航空宇宙センターの輸送システム研究所で作成されたオープンソースのシミュレータであり,各車両の減速値や車線変更などの車両走行情報を取得できる.
SUMO is an open-source simulator created by the Institute of Transportation Systems at the German Aerospace Center.
%
\subsection{Road model}
\label{sec:road_model}
%
% 本研究では,ITSスポットが設置されている高速道路を想定し,道路全長4000 m,片側3車線の一方通行のモデルとし,その間の1000 mを評価区間とした.
This thesis used a one-way model with a total road length of 4000 m and three lanes in one way, assuming an expressway with ITS spots installed as shown in the figure.
%
% 実環境に近い検証を行うため,モデルは道路幅は車線ごとに3.5 m,制限速度を80 〜 100 km/hに設定し,左車線を最優先走行車線とした.
To verify the model as it would be in a realistic environment, the road width was set to 3.5 m for each lane, the speed limit was set to 80 to 100 km/h, and the left lane was set as the highest priority lane.
%
\subsection{Vehicle model}
\label{sec:vehicle_model}
%
% 車両モデルは,主にITSスポットが設置されている高速道路の車種区分\cite{syasyukubun}に基づいて設定した.
The vehicle model was set up based mainly on the vehicle type classification of the expressway on which the ITS spots are installed \cite{syasyukubun}.
%
% 各車両を全長3.5 m,幅1.4 m,最高速度100 km/hと設定し,道路の0 m地点から毎秒1台の車両が任意の車線から発生するようにした.
Each vehicle was assumed to be 3.5 m long, have a maximum speed of 100 km/h, and was generated from the 0 m point on the road.
%
% 前提として,すべての車両はETC2.0車載器を搭載し,評価区間内のすべての車両の車両走行情報を取得できるものとする.
All vehicles are assumed to be equipped with ETC2.0 onboard equipment capable of acquiring vehicle travel information.
%
\subsection{Overview of Vehicle Driving Information}
\label{sec:overview_of_vehicle_driving_information}
%
% 車両走行情報は一般財団法人道路新産業開発機構で規定されてい車載器仕様に沿ったものとなっている.
The vehicle driving information is in accordance with the OBE specifications defined by the New Road Industry Development Organization\cite{probdata_format}.
%
% この仕様によると,車両走行情報は主に走行履歴 (時刻,経度緯度),挙動履歴 (前後加速度,左右加速度,ヨー角速度)で構成される\cite{probdata_format}.
According to this specification, vehicle running information mainly consists of running history and behavior history.
%
\par
%
% 走行履歴は,車両の位置 (緯度・経度),時刻,道路種別等のデータであり,前回記録した地点から200 m走行時または進行方位が45度以上変動した時点で記録される.
The driving history is data such as the vehicle's position (latitude and longitude), time, and road type.
%
It is recorded when the vehicle has traveled 200 m from the last recorded point or when its direction of travel has changed by 45 degrees or more.
%
% 挙動履歴は,車両の位置 (緯度・経度),方位,道路種別,前後加速度,左右加速度,ヨー角速度のデータであり,前後加速度,左右加速度,ヨー角速度のいずれかが表\ref{tab:behavior_history_threshold}に示す閾値を超えたときに記録される.
The behavior history consists of the vehicle's position (latitude and longitude), direction, road type, front/rear acceleration, left/right acceleration, and yaw angular velocity data.
%
It is recorded when either the front/rear acceleration, left/right acceleration, or yaw angular velocity exceeds the threshold value shown in table \ref{tab:behavior_history_threshold}.
%
% よって,路上障害物発生に伴う車両の減速により記録される挙動履歴のデータを用いて,車両の走行状況を把握し,路上障害物検知が可能であると考えられる.
Therefore, it is considered possible to detect obstacles on the road by using the data of the behavior history recorded by the vehicle's deceleration due to the occurrence of obstacles on the road to understand the vehicle's driving conditions.
%
\begin{table}[tpb]
  \caption{Behavior history thresholds}
  \label{tab:behavior_history_threshold}
  \begin{center}
  \begin{tabular}{c|c}
    data item & threshold                   \\ \hline \hline
    front/rear acceleration [$\mathrm{G}$] & $-0.25$      \\ \hline
    left/right acceleration [$\mathrm{G}$] & $\pm0.25$    \\ \hline
    yaw angular velocity [$\mathrm{deg/s}$]& $\pm8.5$ \\ \hline
\end{tabular}
\end{center}
\end{table}
%
\par
%
% ETC 2.0車載器では,GPS,加速度センサ,ジャイロセンサを用いて車両走行情報を収集する.
ETC 2.0 on-board equipment collects vehicle driving information using GPS, acceleration sensors, and gyro sensors.
%
% 車両がITSスポットの通信領域(アンテナから手前約20 mから30 m) を通過する際に送信され,送信された車両走行情報は,ETC 2.0車載器から破棄される.
The information is transmitted when the vehicle passes through the communication area of the ITS Spot (approximately 20 m to 30 m in front of the antenna).
%
% 車両走行情報を収集するシステムは図\ref{fig:configuration_of_the_probe_system}に示すように,ETC 2.0車載器とITSスポット,車両走行情報の収集・集計・管理を行うプローブサーバから構成される.
The transmitted vehicle travel information is discarded by the ETC 2.0 on-board equipment.
%
As shown in Figure \ref{fig:configuration_of_the_probe_system}, the system for collecting vehicle driving information consists of ETC 2.0 on-board units, ITS spots, and probe servers that collect, aggregate, and manage vehicle driving information.
%
% 各道路の管理者はプローブサーバーから,各道路の車両走行情報を収集することができる.
Each road administrator can collect vehicle driving information for each road from a probe server.
%
\begin{figure}[tb]
  \begin{center}
    \includegraphics[width=1.0\linewidth]{configuration_of_the_probe_system.pdf}
  \end{center}
  \caption{Configuration of the probe system}
  \label{fig:configuration_of_the_probe_system}
\end{figure}
%
\subsection{Format of vehicle driving information}
\label{sec:format_of_vehicle_driving_information}
%
% 本稿では,車両走行情報の扱いを簡単にするため,図\ref{fig:format_of_vehicle_driving_information}に示すように,道路を20 m単位のセル上に分割し,各セルを通過した車両走行情報を集計した.
In this thesis, to simplify the handling of vehicle driving information, the road is divided into cells of 20 m units and the vehicle driving information that passed through each cell is aggregated, as shown in Figure\ref{fig:format_of_vehicle_driving_information}.
%
\begin{figure}[tb]
  \begin{center}
    \includegraphics[width=1.0\linewidth]{data_preprocessing.pdf}
  \end{center}
  \caption{Data Preprocessing}
  \label{fig:data_preprocessing}
\end{figure}
%
% 計算の手順について説明する.
The calculation procedure is explained below.
%
% まず,時刻$t$における第$x$番目のセル,第$y$車線を通過した第$m$番目の車両の減速値を$a_{x, y}^{(m)}(t)$とする.
First, let $a_{x, y}^{(m)}(t)$ be the deceleration value of the $m$-th vehicle passing through the $x$-th cell, $y$-th lane at time $t$.
%
% 続いて,通過した車両の減速度の総和を計算し,時刻$T_1$から時刻$T_2$において,積分した各セルの減速度を$A_{x, y}[T_1, T_2]$とし,以下の式で定義する.
The sum of the deceleration of the passing vehicles is then computed and the integrated deceleration of each cell from time $T_1$ to time $T_2$ is defined as $A_{x, y}[T_1, T_2]$ with the following formula
%
\begin{equation}
  A_{x, y}[T_1,T_2]=\displaystyle\sum_{m}\int_{T_1}^{T_2} a_{x,y}^{(m)}(t)dt
  \label{eq:formatted_data}
\end{equation}
%
\subsection{Heat map of deceleration values}
\label{seq:heat_map_of_deceleration_values}
%
% 図\ref{fig:deceleration_heatmap}はそれぞれ路上障害物が中央車線の500 m地点にあり,減速値と式(\ref{eq:cumulative_value})に基づき算出しヒートマップを図示したものである.
Figure \ref{fig:deceleration_heatmap} shows the heatmap calculated based on the deceleration value and the formula (\ref{eq:cumulative_value}) with the road obstacle located at 500 m in the middle lane, respectively.
%
% 図より,路上障害物の手前のセルにおいて減速の傾向があることが確認できる.
The figure shows that there is a tendency to decelerate in the cell in front of the roadside obstacle.
%
% これは,路上障害物を検知した車両が減速することで,衝突を回避するための挙動である.
This behavior is intended to avoid a collision by slowing down the vehicle when it detects a roadside obstacle.
%
% したがって,路上障害物の発生箇所および,減速値は相関があり,検知に有効である可能性がある.
Therefore, the location of obstacles on the road and the deceleration value are correlated and can effective for detection.
%
\begin{figure}[tb]
  \centering
  \includegraphics[width=1.0\linewidth]{deceleration_heatmap.pdf}
  \label{fig:deceleration_heatmap}
  \caption{Deceleration heat map}
\end{figure}
%
\section{Obstacle detection algorithm}
\label{sec:obstacle_detection_algorithm}
%
% 前節のシミュレーション結果より,路上障害物の手前で減速と車線変更の傾向が強くなっていることを確認した.
From the simulation results in the previous section, we confirmed that the tendency to decelerate and change lanes in front of roadside obstacles is stronger.
%
% 本節では,ITSスポットを通して各セルに集積された車両走行情報を用い路上障害物検知を行う手法について説明する.
This section describes a method for detecting obstacles on the road using vehicle travel information accumulated in each cell through ITS spots.
%
\subsection{Flow of obstacle detection}
\label{sec:flow_of_obstacle_detection}
%
% 本研究において,ETC 2.0対応車載器搭載車とITSスポットを用いた路上障害物検知の流れについて説明する.
In this thesis, the flow of road obstacle detection using vehicles equipped with ETC 2.0 on-board equipment and ITS spots is described.
%
% 前提として,ある高速道路にITSスポットが二台設置されており,その間に路上障害物が存在する環境を想定する.
The assumption is that two ITS spots are installed on a certain highway, and that there is an obstacle on the road between them.
%
% またETC 2.0車載器は各時刻において自車両の速度や時刻や位置などの車両走行情報を記録可能であり,各車両が路側に設置されているITSスポットの通信可能範囲に侵入すると,自動的にITSスポットに自車両の走行情報を送信可能である.
The ETC 2.0 on-board unit can record vehicle travel information such as speed, time, and location at each time point.
%
It can automatically transmit vehicle driving information to the ITS spot when a vehicle enters the communication range of the ITS spot installed on the roadside.
%
% これらの前提に伴い,車両のプローブ情報を用いた路上障害物の検知の流れについて説明する.
Along with these assumptions, the flow of road obstacle detection using vehicle driving information is described.
%
\par
%
% まず,高速道路を走行する車両は,路上障害物や事故車などの突発的なこと象の発生,急カーブ,自車両の前での急な車線変更等がない限り,急激に減速しないと考えられる.
First, a vehicle traveling on a highway is not expected to decelerate rapidly unless there is an obstacle on the road, an accident, a sharp curve, or a sudden lane change in front of the vehicle.
%
% ここで,ある時刻において路上障害物が発生したとする.
Suppose that a roadside obstacle is encountered at a certain time.
%
% このとき,路上障害物と同一の車線を走行している車両は路上障害物を安全に回避するため,その手前で減速や車線変更をする.
At this time, vehicles traveling in the same lane as the obstacle decelerate or change lanes in front of the obstacle to safely avoid it.
%
% これらの障害物回避の車両走行情報もITSスポットに送信される.
These obstacle avoidance vehicle driving information is also transmitted to the ITS spot.
%
% 一定期間ITSスポットに蓄積された,各車両の走行情報を交通管制センターに送信し,交通管制センターで送信された車両走行情報を解析し,後続車に路上障害物情報を提供する.
The system transmits the vehicle driving information of each vehicle collected at ITS spots for a certain period of time to the traffic control center, which analyzes the transmitted vehicle driving information and provides information about obstacles on the road to following vehicles.
%
\subsection{Existing Methodology and Positioning of this Study}
\label{sec:existing_methodology_and_positioning_of_this_study}
%
% 車両走行情報を利用して障害物を検知する既存手法の1つに機械学習を用いる手法\cite{tadokoro}が提案されている.
One of the existing methods for detecting obstacles using vehicle driving information is the machine learning method \cite{tadokoro}.
%
% この検知手法は,第\ref{sec:format_of_vehicle_driving_information}節で示した手順で道路をセル上に分割し,各セルに累積された減速値などの車両走行情報を用いて検知を行う.
This detection method divides the road into cells according to the procedure described in section \ref{sec:format_of_vehicle_driving_information} and uses the vehicle driving information such as deceleration values accumulated in each cell for detection.
%
% 各セルに,路上障害物の有無について正解ラベルをつけ,教師あり学習を行いモデルを作成する.
Each cell is labeled with the correct answer for the presence or absence of roadblocks, and a model is created through supervised learning.
%
% 作成したモデルに,ラベルが与えられていないセルに累積されている車両走行情報を入力として与え,障害物を検知する手法である.
This method detects obstacles by providing the model with the vehicle running information accumulated in unlabeled cells as input.
%
% これは,事前にITSスポットに集積された正常・異常データを用いて障害物検知モデルを構築している.
The obstacle detection model is constructed using normal and abnormal data accumulated at ITS spots in advance.
%
% しかし,この手法では,異常データを用いて学習を行う必要があり,路上障害物の有無について正解ラベルを人手でつけることは困難である.
However, this method requires training on anomaly data, and it is difficult to manually assign correct labels for the presence or absence of roadblocks.
%
% また,交通流は季節や時間帯など変化が激しいため,モデルを頻繁に更新する必要がある.
In addition, the model needs to be updated frequently because traffic flows are subject to rapid changes, such as seasonality and time of day.
%
% その際には,様々な環境における異常データを揃えモデルを作り直すことが必要となる.
In such cases, it is necessary to rework the model by aligning anomaly data in various environments.
%
\par
%
% そこで本研究では,絶えず変化する交通流に対応した障害物検知モデルを作成する手法を提案する.
Therefore, this study proposes a method to create an obstacle detection model that responds to constantly changing traffic flow.
%
% 教師なし異常検知のアルゴリズムに,One Class SVM\cite{chen2001one}やIsolation Forest\cite{Isolationforest},Autoencoder\cite{bank2021autoencoders}など様々な手法が提案されている.
Various methods have been proposed for unsupervised anomaly detection algorithms, such as One Class SVM \cite{chen2001one}, Isolation Forest \cite{Isolationforest}, and Autoencoder \cite{bank2021autoencoders}.
%
% Autoencoderは,入力されたデータを出力先のノードの数を減らすことによって,データを一度次元圧縮し,重要な特徴量だけを残した後,再度もとの次元に復元処理をするアルゴリズムである.
Autoencoder is an algorithm that compresses the input data once by reducing the number of nodes in the output node, preserves only the important features, and restores the original dimension.
%
% この特徴から,Autoencoderはデータのノイズ除去や異常検知に用いられている.
Because of this feature, Autoencoder is used for data denoising and anomaly detection.
%
% 異常検知としては,正常データのみを学習させ,異常データが入力された場合,正常データを入力した場合よりも再構成誤差が大きくなるため,その誤差を読み取り異常検知を行うことが可能である.
For abnormality detection, only normal data is trained, and when abnormal data is input, the reconstruction error is larger than when normal data is input, so the error can be read to detect abnormalities.
%
% 加えてデータの異常箇所も特定可能であり,宇宙機の異常検知\cite{sakurada}や災害時の被害状況を把握するシステム\cite{kinami}の提案もされている.
In addition, it can also identify anomalies in the data, and has been proposed as a spacecraft anomaly detection system \cite{sakurada} and a disaster damage assessment system \cite{kinami}.
%
% 本研究では,正常データのみでの学習と,異常箇所の特定という観点からアルゴリズムにAutoencoderを採用する.
In this study, we adopt Autoencoder as the algorithm from the viewpoint of learning with normal data only and identifying anomalies.
%


\bibliographystyle{ieicetr}% bib style
\bibliography{myRef}% your bib database
\begin{thebibliography}{99}% more than 9 --> 99 / less than 10 --> 9
\bibitem{}
\end{thebibliography}

%\profile{}{}
%\profile*{}{}% without picture of author's face

\end{document}
